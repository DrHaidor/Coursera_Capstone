\documentclass{cernatlasnote}
\usepackage[colorinlistoftodos]{todonotes}
\usepackage{placeins}
%\setlength{\parskip}{0.5\baselineskip}

\setlength{\parindent}{0em}
\setlength{\parskip}{1em}

\title{New real estate development and metro station closeness analysis}
\author{Georgy Golovanov}
\date{\today}
\draftversion{0.1}

\begin{document}
\maketitle

\begin{abstract}
Modern cities constantly grow. 
New properties are being built at any time and require the urban infrastructure to be consistent.
This analysis is focused on the real estate properties currently being under construction
%in Moscow city 
by clustering them around current 
%and future 
metro stations to reveal perspective areas.
Since the real estate properties development is growing faster than the city infrastructure
the analysis can be useful for real estate marketing.
\end{abstract}

% Make the review table at the bottom of the title page
\vfill
%\makereviewtable
\clearpage

% Short documentes dont always need a Table of Content / Figures / Tables, so comment out what is not needed
\begingroup
\color{black}
\tableofcontents
%\listoffigures
%\listoftables
\endgroup
\pagebreak


\section{Introduction}
\label{into}
Modern cities constantly grow. 
New properties are being built at any time and require the urban infrastructure to be consistent.
This analysis is focused on the real estate properties currently being under construction
%in Moscow city 
by clustering them around current 
%and future 
metro stations to reveal perspective areas.
Since the real estate properties development is growing faster than the city infrastructure
the analysis can be useful for real estate marketing.

Moscow being a modern permanently growing European city is chosen as a subject for the analysis.
It has been growing especially fast since 2000s and now combines both real estate development 
and urban infrastructure improvement.
The pattern of the development includes construction activities in suburban areas
as well as reconstruction of former factories and manufacture territories within the central city regions.

Moscow is a 18 million city confined within approximately 15~km from its center and has a radial structure.
Being in a list of the most highly populated cities in the World, Moscow also suffers from the
transportation and traffic problems.
As a consequence the fastest and most popular public transport is the underground train system (Moscow metro)
with very extensive structure.
Since it has a radial structure the metro station density is quite high in city center becoming thin for distant regions.

The main purpose of the analysis is  a study of new real estate developments by clustering them around
metro stations.
The closeness of a metro station becomes crucial for distant and suburban neighborhoods that can be important
for real estate marketing estimates.
%Another particular interest is a prior finding of developing properties that can become a good value after commissioning
%of currently projected metro stations.

The analysis is based on datasets containing currently developing properties and metro station geopositions.
The datasets are described more detailed in Sec.~\ref{data}. 
%The clustering procedure is described in Sec.~\ref{method}.

\section{Data acquisition}
\label{data}

In order to perform the study described above the appropriate datasets have to be used, cleaned and modified.
The analysis is based on the following datasets:
\begin{enumerate}
\item Capital construction objects of Moscow;
\item Moscow metro stations.
\end{enumerate}

%https://op.mos.ru/EHDWSREST/catalog/export/get?id=843935

\small{\url{https://data.mos.ru/opendata/7703742961-obekty-kapitalnogo-stroitelstva/passport?
versionNumber=1&releaseNumber=886}}
\section{Methodology}
\label{method}

\section{Results}
\label{results}

\section{Discussion}
\label{disc}


\section{Conclusions}
\label{concl}

%\begin{figure}[ht]
%\centering
%\includegraphics[width=0.5\textwidth]{images/cernlogo.eps}
%\caption{\label{fig:examplecernlogo} Example of how to include a figure. This works with all sorts of formats, eps, pdf, png.}
%\end{figure}
%
%% this will prevent float objects like figures to be moved past this point in the document.
%\FloatBarrier
%
%
%
%
%
%
%
%
%
%\begin{enumerate}
%\item Everyone loves an enumerated list.
%\item If you prefer bulleted lists, see below.
%\end{enumerate}
%
%Of course there are always use cases for list with enumerations, and lists with bullets only, which is why it is useful to have examples of both.
%
%\begin{itemize}
%    \item Everyone loves a bulleted list.
%    \item If you prefer an enumerated list, see above.
%\end{itemize}
%
%\section{The first section after the introduction}
%Since an example of a ToC is not much fun with only one section, lets make another one and throw in some subsections as well.
%
%
%\subsection{The first sub-section}
%How about structuring the document into more subsections.
%
%\subsection{The second sub-section}
%Tables are just as easy as figures to construct and reference, for example this one here (see Table \ref{tab:exampletable}).
%
%\begin{table}[h]
%\begin{center}
%\begin{tabular}{ |c|c|c|p{0.1\textwidth}| } 
%    \hline
%    \rowcolor{lightgray} 
%    col1 & col2 & col3 & col4\\
%    \hline
%    \multirow{3}{4em}{Multiple row} & cell2 & cell3 & cell4 \\ 
%    \cline{3-4}
%    & cell5 & \multicolumn{2}{c|}{cell6 and cell7} \\
%    \cline{3-4}
%    & cell8 & cell9 & cell10 \\ 
%    \hline
%    cell11 & cell12 & cell13 & cell14 \\ 
%    \hline
%    \multicolumn{4}{|c|}{ Multicolumn} \\
%    \hline
%\end{tabular}
%\end{center}
%\caption{A table is as happy about a caption as a fiugre.}
%\label{tab:exampletable}
%\end{table}
%
%\section{Conclusions}
%If the document purpose calls for conclusions, this would be the place to put them.
%
%\bibliography{references}
%\bibliographystyle{plain}

\end{document}